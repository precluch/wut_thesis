%%%%%%%%%%%%%%%%%%%%%%%%%%%%%%%%%%%%%%%%%%%%%%%%%%%%%%%%%%%%%%%%%%%%%%%%%%
% Template of thesis													 %
%																		 %
% Warsaw University of Technology										 %
% Faculty of physics													 %
% 																		 %
% Author: Patryk Bojarski, pbojarski94@gmail.com						 %
% Version: 2.0															 %
% Last modified: 26.12.2016												 %
%%%%%%%%%%%%%%%%%%%%%%%%%%%%%%%%%%%%%%%%%%%%%%%%%%%%%%%%%%%%%%%%%%%%%%%%%%

\input{preamble.tex}
\graphicspath{{LogosPW/}}
\begin{document}

\thispagestyle{empty}
\center
\includegraphics[scale=1]{logo.jpg}
\center
\includegraphics[scale=1]{praca_dyplomowa.jpg}

\vspace{17mm}

\normalsize
na kierunku <nazwa kierunku> \\ % <<<<<<<<<< name of faculty
w specjalności <nazwa specjalności> \\ % <<<<<<<<<< name of speciality

\vspace{15mm}

\Large 
<temat pracy w języku polskim>  \\ % <<<<<<<<<< title of thesis in polish

\vspace{17mm}

\huge 
<imię i nazwisko dyplomanta>  \\ % <<<<<<<<<< name and surname
\normalsize 
Numer albumu <nr albumu-liczba> \\ % <<<<<<<<<< number of index

\vspace{17mm}

promotor \\
<tytuł/stopień naukowy, imię i nazwisko promotora> \\ % <<<<<<<<<< name and surname your superviser

\vspace{15mm}

WARSZAWA <rok> % <<<<<<<<<< year

% empty page
\newpage
\thispagestyle{empty}
\phantom{Nothing here}
\newpage
\clearpage
\phantom{Here neither}

% abstract in polish
\setcounter{page}{3}
\infostyle{Streszczenie}
\vspace{-1.5cm}
\begin{flushleft}
	Tytuł pracy: <tytuł pracy inżynierskiej po polsku> % <<<<<<<<<< title of thesis in polish
\end{flushleft}
\vspace{0.5cm}
\lipsum[1-4] % <<<<<<<<<< here write your abstract
\vspace{0.5cm}
\noindent \textit{Słowa kluczowe: \\ <tu wpisz słowa kluczowe>} % <<<<<<<<<< key words
\vfill
(podpis opiekuna naukowego) \hfill (podpis dyplomanta)

% empty page
\newpage
\thispagestyle{empty}
\phantom{Nothing here}
\newpage
\clearpage
\phantom{Here neither}

% abstract in english
\setcounter{page}{5}
\infostyle{Abstract}
\vspace{-1.5cm}
\begin{flushleft}
	Title of the thesis: <tytuł pracy inżynierskiej po angielsku> % <<<<<<<<<< title of your thesis in english
\end{flushleft}
\vspace{0.5cm}
\lipsum[1-4] % <<<<<<<<<< here write your abstract
\vspace{0.5cm}
\noindent \textit{Słowa kluczowe: \\ <tu wpisz słowa kluczowe>} % <<<<<<<<<< key words
\vfill
(podpis opiekuna naukowego) \hfill (podpis dyplomanta)

% empty page
\newpage
\thispagestyle{empty}
\phantom{Nothing here}
\newpage
\clearpage
\phantom{Here neither}

\setcounter{page}{7}
\infostyle{Oświadczenie o samodzielności wykonania pracy}
\vspace{-1.5cm}
\begin{flushleft}
	Politechnika Warszawska \\ 
	Wydział fizyki \\
	\vspace{0.5cm}
	Ja niżej podpisany/a: 
\end{flushleft}
\center \textit{\textbf{<Imię i nazwisko, nr albumu>}} % <<<<<<<<<< name, surname and number of index
\justify student/ka Wydziału Fizyki Politechniki Warszawskiej, świadomy/a odpowiedzialności prawnej oświadczam, że przedłożoną do obrony pracę dyplomową inżynierską pt.:
\center \textit{\textbf{<Tytuł pracy dyplomowej>}} % <<<<<<<<<< title of your thesis
\justify wykonałem/am samodzielnie pod kierunkiem
\center \textit{<Tytuł naukowy promotora>} % <<<<<<<<<< name and surname your superviser
\justify Jednocześnie oświadczam, że: \\
\begin{itemize}
	\itemi praca nie narusza praw autorskich w rozumieniu ustawy z dnia 4 lutego 1994 o prawie autorskim i prawach pokrewnych, oraz dóbr osobistych chronionych prawem cywilnym,
	\itemi praca nie zawiera danych i informacji uzyskanych w sposób niezgodny z obowiązującymi przepisami,
	\itemi praca nie była wcześniej przedmiotem procedur związanych z uzyskaniem dyplomu lub tytułu zawodowego w wyższej uczelni,
	\itemi promotor pracy jest jej współtwórcą w rozumieniu ustawy z dnia 4 lutego 1994 o prawie autorskim i prawach pokrewnych.
\end{itemize}
\justify Oświadczam także, że treść pracy zapisanej na przekazanym nośniku elektronicznym jest zgodna z treścią zawartą w wydrukowanej wersji niniejszej pracy dyplomowej.
\vfill
Warszawa, dnia <data> \hfill (podpis dyplomanta) % <<<<<<<<<< date

% empty page
\newpage
\thispagestyle{empty}
\phantom{Nothing here}
\newpage
\clearpage
\phantom{Here neither}

\setcounter{page}{9}
\infostyle{Oświadczenie o udzieleniu Uczelni licencji do pracy}
\vspace{-1.5cm}
\begin{flushleft}
	Politechnika Warszawska \\ 
	Wydział fizyki \\
	\vspace{0.5cm}
	Ja niżej podpisany/a: 
\end{flushleft}
\center \textit{\textbf{<Imię i nazwisko, nr albumu>}} % <<<<<<<<<< name, surname and number of index
\justify student/ka Wydziału Fizyki Politechniki Warszawskiej, niniejszym oświadczam, że zachowując moje prawa autorskie udzielam Politechnice Warszawskiej nieograniczonej w czasie, nieodpłatnej licencji wyłącznej do korzystania z przedstawionej dokumentacji pracy dyplomowej pt.:
\center \textit{\textbf{<Tytuł pracy dyplomowej>}} % <<<<<<<<<< title of your thesis
\justify w zakresie jej publicznego udostępniania i rozpowszechniania w wersji drukowanej i~elektronicznej*.
\\~\\~\\~\\~\\~\\~\\~\\~\\~\\~\\~\\~\\~\\~\\~\\~\\~\\~\\~\\
Warszawa, dnia <data> \hfill (podpis dyplomanta) % <<<<<<<<<< date

\begin{center}
	\color{plum}
	\line(1,0){460}
\end{center}
\setstretch{1}
\footnotesize \noindent $^{*}$Na podstawie Ustawy z dnia 27 lipca 2005 r. Prawo o szkolnictwie wyższym (Dz.U. 2005 nr 164 poz. 1365) Art.~239. oraz Ustawy z dnia 4 lutego 1994 r. o prawie autorskim i prawach pokrewnych (Dz.U. z 2000 r. Nr 80, poz. 904, z późn. zm.) Art. 15a. "Uczelni w rozumieniu przepisów o szkolnictwie wyższym przysługuje pierwszeństwo w~opublikowaniu pracy dyplomowej studenta. Jeżeli uczelnia nie opublikowała pracy dyplomowej w ciągu 6 miesięcy od jej obrony, student, który ją przygotował, może ją opublikować, chyba że praca dyplomowa jest częścią utworu zbiorowego."
\setstretch{1.15}

% empty page
\newpage
\thispagestyle{empty}
\phantom{Nothing here}

\end{document}